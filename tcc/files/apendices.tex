\setlength{\absparsep}{18pt} 
\begin{apendicesenv}

\chapter{Códigos}
\label{apend-code}

Este apêndice contém os algoritmos, códigos, \textit{scripts} e funções utilizadas para o desenvolvimento do trabalho que são citadas o longo do decorrer da dissertação. O conjunto completo dos arquivos de código se encontra no diretório \verb|/src| do repositório do \textit{Github} <\url{github.com/GDevigili/TCC-IBGE}>.

    \section{Algoritmo de \textit{Unnesting}}
    \label{apend-unnesting}

    O presente algoritmo realiza o processo de \textit{unnesting}, que visa o nivelamento de uma estrutura de dados em nível único, duplicando dados quando necessário. O processo de desaninhamento sobe os dados de um valor ``filho'' (de grau inferior) até o nível de seu valor ``pai'' (de grau imediatamente superior) de forma recursiva. 

\begin{lstlisting}[label={lst:unnesting},language=Python, caption=Algoritmo de \textit{unnesting} em \textit{python}.]
def unnest_json(json_dict: dict, col_name: str = '') -> dict:
    output = {}
    def flatten(column, col_name:str = ''):
        # Se a coluna é do tipo dict, continua a recursão
        if type(column) == dict:
            for key in column:
                # Determina o nome da coluna para cada chave (key)
                if col_name == '':
                    new_col_name = str(key)
                else:
                    # adiciona o pai da coluna ao seu nome
                    parent_col = col_name.split('_')[-1]
                    new_col_name = parent_col + '_' + str(key)
                # Chama recursivamente a função flatten para a chave atual
                flatten(column[key], new_col_name)
        else:
            # Se a coluna não é do tipo dict, salva o valor dela no output
            output[col_name] = column
    flatten(json_dict, col_name)
    return output
\end{lstlisting}


\section{Algoritmo de conversão de \textit{string} JSON para um objeto do tipo \textit{pandas.DataFrame}}

    O \textit{script} abaixo converte um objeto do tipo \textit{string} contendo um código JSON para um objeto do tipo \verb|DataFrame| da biblioteca \verb|pandas|, para tal, utilizando a função de \textit{unnesting}, presente em \textit{listing} \ref{lst:unnesting}. Um exemplo de uso é a conversão do exemplo em \textit{listing} \ref{lst:exemplo-api-localidades} da sessão \ref{metodoslogia-API} para a tabela \ref{tab:exemplo-api-localidades}.

\begin{lstlisting}[label={lst:make-df},language=Python, caption=Algoritmo de transformação de \textit{string} JSON para um objeto do tipo \textit{pandas.DataFrame}.]
import pandas as pd

def make_df(json_dict: dict) -> pd.DataFrame:
    """Return a DataFrame from a json dict.

    Args:
        json_dict (dict): A dict containing the json data.

    Returns:
        pd.DataFrame: A DataFrame with the data.
    """
    # Create a list to store the rows
    rows = []
    # Iterate over each item on the json dict
    for item in json_dict:
        # Unpack the json dict into a single row dict
        row = unpack_json(item)
        # Add the row to the list
        rows.append(row)
    # Create a DataFrame from the rows list
    df = pd.DataFrame(rows)
    return df
\end{lstlisting}

\section{Adaptador HTTP customizado}

    Devido ao erro \verb|UNSAFE_LEGACY_RENEGOTIATION_DISABLED| encontrado ao tentar realizar requisições às APIs do IBGE utilizando sistemas operacionais baseados em \textit{Linux} (foi utilizado Ubuntu 20 nos testes), foi necessário criar um adaptador customizado do \textit{Hypertext Transfer Protocol} (HTTP) utilizando um contexto legado do protocolo \textit{Secure Sockets Layer} (SSL).

\begin{lstlisting}[label={lst:http-adapt},language=Python, caption=Algoritmo de transformação de \textit{string} JSON para um objeto do tipo \textit{pandas.DataFrame}.]
import requests
import urllib3
import ssl

class CustomHttpAdapter (requests.adapters.HTTPAdapter):
    """Transport adapter that allows us to use custom ssl_context."""
    def __init__(self, ssl_context=None, **kwargs):
        self.ssl_context = ssl_context
        super().__init__(**kwargs)

    def init_poolmanager(self, connections, maxsize, block=False):
        self.poolmanager = urllib3.poolmanager.PoolManager(
            num_pools=connections, maxsize=maxsize,
            block=block, ssl_context=self.ssl_context)

def get_request():
    """Return a requests session with a custom SSL context."""
    ctx = ssl.create_default_context(ssl.Purpose.SERVER_AUTH)
    ctx.options |= 0x4  # OP_LEGACY_SERVER_CONNECT
    session = requests.session()
    session.mount('https://', CustomHttpAdapter(ctx))
    return session

def get_request_json(url: str)-> dict:
    """Return a json from a request."""
    response = get_request().get(url)
    return response.json()
\end{lstlisting}

\section{\textit{Script} para a carga de indicadores da API de países}

    O código a seguir realiza requisições para a API de países do serviço de dados do IBGE, retornando um conjunto de dados com 34 indicadores de 193 países. O módulo \verb|utils| importado no \textit{script} está no diretório \verb|src| do repositório do \textit{Github}. As funções utilizadas são \verb|make_df()| e \verb|get_request_json()| (\textit{listings} \ref{lst:make-df} e \ref{lst:http-adapt}, respectivamente).
    
\begin{lstlisting}[label={lst:api-paises},language=Python, caption=\textit{Script} de carga dos indicadores da API de países.]
import pandas as pd
from utils import *

url_paises = "https://servicodados.ibge.gov.br/api/v1/paises/"
df_paises = make_df(get_request_json(url_paises))
df_paises = df_paises.rename(columns={
    'id_M49': 'id-pais'
    ,'id_ISO-3166-1-ALPHA-2': 'sigla-2'
    ,'id_ISO-3166-1-ALPHA-3': 'sigla-3'
})
url_indicadores = "https://servicodados.ibge.gov.br/api/v1/paises/indicadores/"
df_indicadores = make_df(get_request_json(url_indicadores))

country_list = df_paises['sigla'].unique()
indicator_list = df_indicadores['id'].unique()

df_country_indicators = pd.DataFrame()
for country in country_list:
    # making the request
    url_country = f"https://servicodados.ibge.gov.br/api/v1/paises/{country}/indicadores"
    df_country = make_df(get_request_json(url_country))
    # unpacking the lists
    # take the series dict out of the list (there will be always one or zero values on the list)
    df_country = df_country.explode('series')
    # get the country code (e.g. BR) and country name from the series country and concat it into each of the df_country lines
    df_country = pd.concat([df_country, pd.json_normalize(df_country['series'])], axis = 1)
    # drop the column series as it won't be used anymore
    df_country = df_country.drop('series', axis = 1)
    # unpack the values for each date
    df_country = df_country.explode('serie')
    df_country['ano'] = df_country['serie'].apply(lambda x: list(x.keys())[0] if type(x) == dict else None)
    df_country['valor_indicador'] = df_country['serie'].apply(lambda x: list(x.values())[0] if type(x) == dict else None)
    df_country = df_country.drop('serie', axis = 1)
    df_country_indicators = pd.concat([df_country_indicators, df_country], ignore_index=True)
\end{lstlisting}

\section{Leitura de arquivo \textit{Stata} no \textit{python}}

    Para realizar a leitura de um arquivo de extenção \verb|.dta|, às vezes se faz necessário carregar ele como um objeto do tipo \textit{StataReader} da biblioteca \verb|pandas.io.stata|. 

    No primeiro bloco de código, a variável \verb|data_micro| recebe os dados no formato de microdados lidos pelo \textit{reader}, \verb|col_labels| os nomes de coluna e \verb|value_lables| os valores identificados pelos IDs que estão nos microdados. Tendo isto em mãos, os valores são substituídos no laço de repetição para obter-se o \textit{DataFrame} final.

\begin{lstlisting}[label={lst:read-stata},language=Python, caption=\textit{Script} de carga dos indicadores da API de países.]
# Read the file and the labels
with StataReader('../../dados/microdados-processados/amostra_domicilios_2010_RJ.dta') as reader:
    data_micro = reader.read(convert_categoricals=False)
    col_labels = reader.variable_labels()
    value_labels = reader.value_labels()

# Remove '_lbl' from the keys in value_labels
value_labels = {k.replace('_lbl', ''): v for k, v in value_labels.items()}

# Apply the labels to the data
for col, labels in value_labels.items():
    if col in data_micro:
        data_micro[col] = data_micro[col].map(labels)

# Apply the labels to the columns
data_micro = data_micro.rename(columns=col_labels)
\end{lstlisting}

\chapter{Visualização dos dados}

\section{Mapa do percentual de domicílios com acesso à água encanada no Estado do Rio de Janeiro - Dados Agregados}
\label{sec-ap-img}

    Mapa análogo ao da figura \ref{fig:mapa-rj} que tem como fonte de dados os agregados da API. O mapa também é utilizado para comparação na  figura \ref{fig:mapa-rj-compara}.

    Os dados coletados da API são os referentes à variável 100009: ``Existência de água canalizada'' em unidades percentuais. A variável foi filtrada para apenas conter valores referentes ao Estado do Rio de Janeiro agregados pelo nível de município (N6). A seguinte URL redireciona para o JSON contendo os dados: \url{https://servicodados.ibge.gov.br/api/v3/agregados/2065/periodos/-6/variaveis/96%7C1000096?localidades=N1[all]|N13[330101]|N6[N3[33]]&classificacao=471[0,12219]}.

\begin{figure}[ht]
    \centering
    \includegraphics[width=\textwidth]{files/img/mapa água agreg.png}
    % \caption{Comparação entre os mapas de percentual de domicílios com água canalizada no município do Rio de Janeiro. Na esquerda o gráfico com base nos microdados e na direita com base nos dados agregados da API. Fonte: Dados do Autor (2023).}
    \caption{Mapa do percentual de domicílios com água encanada no Estado do Rio de Janeiro. Fonte: Dados do Autor (2023)\protect\footnotemark}
    \label{fig:mapa-rj-agg}
\end{figure}
\footnotetext{Com base nos dados agregados do Censo Demográfico de 2010.}


\end{apendicesenv}